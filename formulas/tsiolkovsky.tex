\documentclass[../formulas.tex]{subfiles}
\begin{document}
\begin{equation}
	v-v_0= -c\ln\left ( \frac{m(t)}{m_0}\right ) - gt
\end{equation}
\begin{itemize}
	\item $t$ being the time since some moment
	\item $v$ being the velocity of a rocket
	\item $v_0$ being the inital velocity of the rocket
	\item $c$ being the velocity the propellant is exhausted at
	\item $m_0$ being the initial mass of the rocket
	\item $m(t)$ being the mass of the rocket at time $t$
	\item $g$ being the gravitational acceleration
\end{itemize}
The formula relates the change in velocity of the rocket to the amount of propellants and at
what velocity they're exhausted from the nozzle.
It is ideal in the sense that effects like drag, the changing atmospheric pressure and changing
gravitation are neglected here.\\
It is especially useful as it allows for example to examnine the maximum change in velocity,
here called $\Delta v$ the vehicle can reach with a given exhaus velocity, inital and
final mass ($m_i$)
\begin{displaymath}
	\Delta v = -c\ln\left ( \frac{m_f}{m_0}\right ) - gt
\end{displaymath}
\end{document}
